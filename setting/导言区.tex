\usepackage{xeCJK}%设置全局字体
\usepackage{fontspec}%设置英文字体(包括公式字体)
\usepackage{amssymb}%\mathbb等公式字体命令
\usepackage[tbtags]{amsmath}
\usepackage[thmmarks,amsmath]{ntheorem}%证明过程环境
\usepackage{mathtools}%multlined公式环境
\usepackage{geometry}%页面边距
\usepackage{fancyhdr}%自定义页眉页脚
\usepackage{setspace}%定义行间距,不是段间距
\usepackage[bookmarks=false]{hyperref}%超链接

%\setCJKmainfont{SimSun}[BoldFont=SimHei, ItalicFont=KaiTi]
\let\heiti\relax%清除旧定义
\let\fangsong\relax
\let\songti\relax
\newCJKfontfamily{\heiti}[AutoFakeBold={3.17}]{SimHei}%重定义\heiti
\newCJKfontfamily{\fangsong}[AutoFakeBold={3.17}]{FangSong}%重定义\fangsong
\newCJKfontfamily{\songti}[AutoFakeBold={3.17}]{SimSun}%重定义\songti

\renewcommand{\theequation}{%重新定义公式编号格式
	\thesection.\arabic{equation}}
\numberwithin{equation}{section}%每一节单独编号

\geometry{screen}
\hypersetup{%超链接格式
	unicode=true,%在acrobat中pdf书签允许有latin符号(xelatex只允许true)
	pdftoolbar=false,%acrobat工具栏
	pdfmenubar=true,%acrobat菜单栏
colorlinks=false,%不用彩色超链接
%bookmarks=false,%不制作书签,放到宏包属性中
%bookmarksopen=false,%书签不自动打开
%bookmarksnumbered=false,%书签不标号
pdfborder=000,%无超链接边框
pdfpagemode=UseNone,%FullScreen全屏显示;UseThumbs显示缩略图;UseOutlines显示书签;UseOC显示图层;UseAttachments显示附件
pdfstartview=Fit%适合页面
}
\geometry{a4paper,left=25mm,right=25mm,top=25mm,bottom=25mm}
\setlength{\voffset}{-10mm}                        
\setlength{\topmargin}{0mm}
\setlength{\headheight}{5mm}
\setlength{\headsep}{5mm}
\setlength{\footskip}{7.5mm}

\fancypagestyle{NoHeader}{%无页眉页面风格
	\fancyhf{}%切换页面风格
	\pagenumbering{arabic}%设置页码格式,阿拉伯数字标页
\fancyhead[C]{}%中间页眉
	\renewcommand{\headrulewidth}{0pt}%分隔线宽度0磅
	\fancyfoot[C]{\zihao{-5}\setmainfont{Times New Roman}\thepage}%中间页脚
	\renewcommand{\footrulewidth}{0pt}
}

{%定理类环境有编号风格
	\theoremstyle{plain}
	\theoremheaderfont{\bfseries}
	\theorembodyfont{\normalfont}
\newtheorem{definition}{\indent 定义}[section]
\newtheorem{theorem}{\indent 定理}[section]%定理
\newtheorem{proposition}[theorem]{\indent 命题}
\newtheorem{lemma}[theorem]{\indent 引理} 
\newtheorem{corollary}[theorem]{\indent 推论}
\newtheorem{example}{\noindent 例题}
\newtheorem{thinking}{\noindent 思考题}
}
{%注的环境配置
	\theoremstyle{nonumberplain}
	\theoremheaderfont{\bfseries}
	\theorembodyfont{\normalfont}
	\newtheorem{remark}{\indent 注:}
}
%\qedsymbol{\ensuremath{_\blacksquare}} %%在如无需证明的推论等定理类环境中使用 \qed 以显示证明结束符, 此处使用 \ensuremath{_\blacksquare} 而不是 $\square$ 以便在以数学环境结束的证明中也可以正常使用.
\qedsymbol{\ensuremath{_\Box}}%空心方块
{%证明过程环境配置
\theoremstyle{nonumberplain}
\theoremheaderfont{\bfseries}
\theorembodyfont{\normalfont}
%\theoremsymbol{\ensuremath{_\blacksquare}}%放进盒子的符号
\theoremsymbol{\ensuremath{_\Box}}%放进盒子的空心方块
\newtheorem{proof}{\indent 证明}
}

\renewcommand{\theenumi}{\arabic{enumi}}
\renewcommand{\labelenumi}{(\theenumi)}%编号使用"(1)"格式

\allowdisplaybreaks[2]%长公式断页
%	\bibliographystyle{unsrt}%参考文献排版风格
%\renewcommand{\baselinestretch}{1.5}%1.5倍行距
\setlength{\lineskiplimit}{2.625bp}%五号字1/4高
\setlength{\lineskip}{2.625bp}%避免两行过于紧凑
\setlength{\parskip}{0pt}%段间距